\chapter{Summary and Future Work}
\label{cha:summery_and_future_work}
In this chapter, we show possible future work on the simulation and our multi-robot system for LLSF in section~\ref{sec:future_work}. Afterwards, we conclude the thesis in section~\ref{sec:summery}.

\section{Future Work}
\label{sec:future_work}
In the near future, the simulation and our multi-robot system solution for LLSF need to be changed because LLSF is going to be changed~\cite{llsf_changes}. At the RoboCup 2014, it is likely that two teams play at the same time on the same field with double size. The machines with RFID-readers might also get replaced by small assembly machines. This also demands hardware changes on the Robotino, such as adding a new gripper. We probably also add different sensors, such as a stereoscopic camera to the Robotino to get a more reliable puck and light-signal detection. The simulation and the agent have to be adapted to these changes.\\
The simulation can also be extended to work in other domains with other robots, such as domestic service robots. The motor and basic sensors we implemented can be reused for this. To improve the expandability of the simulation even further it is also useful to refactor the Gazebo plugins. The existing modules for sensors and actuators can be separated into plugins and configuration values can be put in separate configuration files which are read at the start of the plugins. This would allow reconfiguration and building of new simulated robots without recompiling. Two other small improvements we want to implement in the future are the simulation of the omnivision-camera and providing the simulation time in ROS for more realistic \texttt{move\_base} results.\\
The simulation developed in this thesis can be the foundation of a variety of future projects. For example, the the visualization possibilities could be used to visualize agent belief and intention, what would simplify the development of more advanced agent strategies, or the automatic simulation runs can be used to implement some kind of machine learning or parameter finding. The simulation is an important tool for future development for the LLSF. In the near future, we have to solve the problems we discovered in the section~\ref{sec:multi_robot_strategies}. This includes navigation-coordination with multiple robots to avoid that two robots block each others path, improving the locking approach to be more robust against failure, increasing reliability of low level skills and recovery from situations shown in Figure~\ref{fig:fails}. Furthermore, we have to develop large agent improvements, such as a more flexible and optimized task allocation strategy and producing without waiting at the machine until the production is finished to do other tasks in the meantime.

\section{Summary}
\label{sec:summery}
In this thesis, we have developed a multi-robot simulation for the LLSF. The simulation was designed to allow for efficient testing of multi-robot coordination and and other important parts of multi-robot systems. We achieved this by implementing a realistic LLSF simulation in Gazebo and connecting this simulation to Fawkes. In Fawkes, simulation plugins exchange real sensor and actuator plugins and provide the same interfaces so that the components we want to test can operate in the same way as when running on a real robot. To increase efficiency during testing and evaluation, we implemented start-up scripts and automated test runs with statistics and reconstructions of simulated runs. The simulation also was designed and implemented in a way so that future changes on the LLSF and the Carologistics Robotino are easy to implement in the simulation. This also allows using the simulation in other domains and with other robots because the small plugins and modules we implemented can be reused. For better testing possibilities, we included multi-level abstraction which allows us to test high level components with either simulated sensor data or ground truth information. \\
To improve our multi-agent system and test the evaluation capabilities of the simulation, we implemented a dynamic role change to adapt the task allocation to the situation and recycling to score more points more continuously during the production of complex products. Furthermore, we extended our previous role-based approach to work with three instead of two robots in the exploration and production phases.\\
We evaluated how realistic the simulation is. There are limitations of the simulation realism mainly caused by missing visual phenomenon, such as realistic reflections, and defining realistic physical properties, such as friction parameters. Within these limitations, we can realistically simulate the LLSF environment and sensors and actuators of the Robotino, what is more than sufficient for the purpose of evaluating multi-robot systems.\\
In the evaluation, we also analyzed several configurations with different role assignments and agent improvements and compared them with the performance of the previous system. We found that two $P_3$ agents perform significantly better than other configurations because producing $P_3$ pucks continuously provides many points is has the least likelihood to fail. Recycling improved the performance of the system because it provides points early in the production process of a complex product, but also added more steps to the production process what increased the likelihood to fail at some point during the production. The dynamic role change provided a smaller advantage than we expected because role change happens only rarely and late. The performance of all configurations was limited by the several problems we identified in the evaluation. These problems mainly include navigation coordination, reliability of low level movement and failure handling. We noticed that these problems have to be solved before improvements on the high level agent can use their full potential. These problems especially limit the performance of configurations with three robots as we showed in the evaluation.\\
In the previous section about future work, we described that the simulation is extendable for future changes and can be the foundation of several future projects. We showed that the simulation is extendable and useful in the testing process during the Bonding Hackathon. The simulation is also an important tool for future developments on our solution for the LLSF. The simulation has already been successfully used in some developments beside agent improvements. Two of these developments are the use of an alternative navigation and collision avoidance plugin on the Robotino and the vision detection of colored pucks for the Hackathon.
