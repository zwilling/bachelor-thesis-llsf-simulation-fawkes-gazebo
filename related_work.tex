\chapter{Related Work}
In this chapter we present work related to this thesis. In section 3.1, we descibe alternatives to Gazebo and why we choose Gazebo. In section 3.2, we give an overview of other popular simulations, especially a simulation that also uses Gazebo and other multi-robot simulations, and in section 3.3, we describe our current multi-robot strategy as well as other attractive strategies our simulation should be able to evaluate. In the following, we will use the terms \textit{multi-robot} and \textit{multi-agent} interchangeably because the common term in litrature is multi-agent and we want to use multi-robot. A multi-agent system can consist of fully abstract agents, such as \textcolor{red}{...}, whereas we are dealing with multi-robot system and therefore robots that interact physically with their environment.


\section{Other Simulators}
\textcolor{red}{blabla?}
\subsection{Stage}
\textit{Stage}\footnote{\url{http://playerstage.sourceforge.net/index.php?src=stage}} is an Open Source multi-robot simulator for a two dimensional environment. It is a part of the \textit{Player/Stage} project~\cite{{PlayerStage}}. \textit{Player} is a robot control server, which provides a simple connection between a robot control program and the sensors and actuators of a robot. Stage uses this connection and replaces the sensors and actuators by simulated ones. Stage was popular simulator in the last years and is able to simulate a large amount of robots. For the simulation, stages simple and computationaly cheap models, that provide still enough fidelity for most applications. Furthermore the computational effort is linear in the number of robots. Because of that, Stage can simulate large amounts of robots~\cite{stage_massive}. This makes Stage attractive for the simulation of swarms and multi-robot systems with many robots.

\subsection{Webots}
\textit{Webots}\footnote{\url{http://www.cyberbotics.com}} is a commercial 3D-simulator~\cite{{Webots}}. It is platform independent features a whole development environment for robot software including an editor and an \textcolor{red}{API} for inter-robot communication. Webots supports many programming languages and supports a variat of standard robots and sensors out of the box. It is also able to simulate multiple robots and provides a ROS and \textcolor{red}{Matlab} interface.\\
Webots was used in a RoboCup Soccer simulation league~\cite{webots_robocup}.

\subsection{USARSim}
\textit{USARSim} is a 3D-simulator that was developed for urban search and rescue scenarios~\cite{USARSim}. It was envolved to be able to simulate other domains too. Therefore, the abbriviation USARSim now stands for ``Unified System for Automation 
and Robotics Simulation'' instead of te previous ``Urban Search and Rescue Simulation''~\cite{usarsim_new}. It is platform independent and free of charge for research and education. It's advantages are the graphical quality of the Unreal engine\footnote{\url{http://www.unrealengine.com/udk}} and the PhysX physics engine\footnote{\url{https://developer.nvidia.com/physx}}. USARSim also provides an interface to ROS~\cite{USARSimROS} and is able to simulate multiple robots.\\
USARSim is used as a basis for the RoboCup Rescue Simulation League we also present in section 3.2.2.

\subsection{SimSpark}
\textit{SimSpark}\footnote{\url{http://simspark.sourceforge.net/}} is an Open Source 3D simulator~\cite{simspark_old}. It was developed by the RoboCup community for the RoboCup 3D Soccer Simulation and has been used there since 2004. Therefore SimSpark is able to simulate multiple robots and specialised on the NAO as soccer robot. There are several improvementsand new features of SimSpark~\cite{SimSpark,Visualization}, such as visualization of agent-intentions, realistic servo motors for the NAO and better support of heterogenous robot teams.

\subsection{Robotino Sim Professional}
\textit{Robotino Sim Professional}\footnote{\url{http://www.festo-didactic.com/int-en/learning-systems/software-e-learning/robotino-sim-view/robotino-sim-professional.htm}} is a 3D-simulator for the Robotino developed by its manufacturer Festo. It is a commercial software and only usable on Microsoft Windows. Furthermore, it is limited to the Robotino and the default sensors the robotino is sold with.

\subsection{Choice for Gazebo}
We aready presented Gazebo and some of its advantages in section 2.4. Here we explain why we have chosen Gazebo instead of an alternative we described above.\\
The stage simulator is well suited for multi-robot systems and comparatively fast and easy to use. However, the restriction to dimensions is a problem. We want to simulate vision components as well to be able to test our system as a whole and more realistic. We also want to be extendable for future changes of LLSF or the use of the simulator in an other domain, such as the RoboCup @Home league.\\
Webots is a proven simulator with many features, but has the same problems as Robotino Sim Professional. On the one hand both simulators are commercial and on the other hand both are no Open Source software. Therefore it can be difficult to expand the simulation how we need it.\\
SimSpark is not so well suited as Gazebo because it and the community behind are mostly specialized on the RoboCup Soccer Simulation. Gazebo is used in large variaty of domains and has a bigger community. Gazebo is well founded by the Open Source Robotics Foundation\footnote{\url{http://osrfoundation.org/}} and Willow Garage\footnote{\url{http://www.willowgarage.com/}} and there are also many plans to develop Gazebo further\footnote{\textcolor{red}{link roadmap}}. Therefore Gazebo is likely to become even more important in the future.\\
An other important argument for Gazebo is that it was used at KBSG before. It was used as a simulator for MSL~\cite{MultiLevelAbstraction} and it was used for a scene reconstruction~\cite{KlingenDA}. We will present both later in this chapter.\\
There are only few disadvantages of Gazebo for this thesis. A disadvantage is that Gazebo can not handle larger number of robots. The complexity of the simulations limits the simulation speed when using multiple robots. The number of robots to simulate at a reasonable speed is in the order of ten~\cite{GazeboDesign}. This can become a problem if the simulation runs on a slow computer or there are more robots to simulate at the same time.

\section{Simulations}
\textcolor{red}{blabla?}
\subsection{Virtual Robotics Challenge}
The \textit{Virtual Robotics Challenge (VRC)} is a competition by DARPA\footnote{DARPA is the Defense Advanced Research Projects Agency of the USA.}. The goal of the competitin is to solve challenging tasks with a humaniod robot in a simulation. VRC is the first part of the \textit{DARPA Robotics Callenge (DRC)}\footnote{\url{http://www.theroboticschallenge.org/}}. DRC aims to spur the development of robots that can operate in a desaster scenario if the situation is too dangerous for humans. An example for such a scenario is the desaster in the Fukushima nuclear power plant after tsunami in March 2011~\cite{fukushima}. The developed robots should be able to operate in environment made for humans, even if the environment is damaged, and to use tools made for humans, such as screwdrivers and cars. During the competition the robots act partly autonomously and are supervised by a human instructor. Altough the technology fostered by DRC is important and useful without doubt, it seems questionable what DARPA as a military agency will use this technology for.\\
VRC is related to this thesis because it also uses the Gazebo simulator~\cite{IEEESpectrum}. Therefore it shows what Gazebo is capabile of and how important a simulation is for the development and research.


use simulation cutting edge tech dev
capabilities of gazebo, processing power
related to this thesis because
rules
held
\subsection{RoboCup Simulation Leagues}
\subsection{Scene Reconstruction}
\textcolor{red}{sectionname?}
\subsection{Simulation Environment for the Middle Size League}

\section{Multi-Agent Strategies}
\subsection{Incremental Task-level Reasoning}
\subsection{Marked Based Strategies}
