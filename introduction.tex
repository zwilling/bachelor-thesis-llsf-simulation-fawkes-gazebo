\chapter{Introduction}

Multi-robot systems are combinations of multiple robots that are made to work together on specific tasks. Currently, they are mostly used in assembly lines, where many robot-arms simultaneously do repetitive tasks with the same object. More autonomous examples of multi-robot systems can be found in the warehousing domain. Here, many robots bring demanded goods and store new ones~\cite{Kiva}. The major advantages of multi-robot systems are high flexibility and the possibility to run different tasks in parallel. The amount of possible applications is large. Besides assembly and warehousing, they also can be used in rescue scenarios~\cite{mas_rescue}, soccer~\cite{mas_soccer}, planetary exploration~\cite{mas_space}, logistics and more.\\
Developing a robotic system is challenging. Robots have to detect objects, localize themselves, reason about their surrounding and manipulate it. Developing a multi-robot system is even harder because the robots have to do all this in coordination with other robots. Furthermore the systems have to be robust against the failure of single robots and should be as efficient as possible. Because of these challenges, testing is an essential part of the development process. It is necessary to identify mistakes in the source code and to evaluate how the system performs in a complex environment. Often, tests reveal problems the developer has not expected. However, testing can be difficult and time consuming. The robot and the environment have to be available and set up. A component to test can depend on other components that are still in development. Some tests require cautious execution because the robot could harm itself or the environment and full system tests have to run a longer time. Testing a multi-robot system is even harder. On the one hand the testing effort scales with the number of robots. On the other hand the system is more complex and therefore requires more test runs for evaluation.\\
In this thesis we tackle these problems by developing a multi-robot simulation environment. We simulate multiple robots in a physically and visually realistic three-dimensional environment. The robot software runs as in the real world and gets simulated sensor data instead of real sensor data. When the robot software uses actuators, the actions are executed in the simulation. This brings many advantages and can speed up the testing process. With a simulation environment, the majority of tests does not need available robots and the environment, the setup can be automated, unfinished components other components depend on can be simulated as well. Furthermore we want to evaluate the whole multi-robot system by measuring its performance in multiple runs. This makes us able to easily compare different configurations and strategies in the simulation.\\
Such a simulation is useful in most robot developments. In this thesis we concentrate on the logistics domain and the mobile robot platform \textit{Robotino}. We participate with the \textit{Carologistics} team in the \textit{Logistic League sponsored by Festo (LLSF)}. The Carologistics is a joint team consisting of the Knowledge-based Systems Group at RWTH Aachen University, the IMA/ZLW \& IFU Institute Cluster at RWTH Aachen University and the Department for Electrical Engineering and Information Technology, Robotics Group at FH Aachen. LLSF is an industrial motivated competition within the RoboCup initiative. Three robots have to manage the material flow in a production area. The goal is to produce as many ordered products by feeding different machines in the production area with resources and intermediate products. To achieve a good performance, robust robot behavior and an efficient scheduling is necessary. Because of this, we have a special need for a multi-robot simulator, which can test the performance of single robots as well as the efficiency of the whole multi-robot system.\\
We use the \textit{Fawkes} robot software framework to control the robots and we have chosen \textit{Gazebo} as robot simulator. Important task of this thesis are the connection between Fawkes and Gazebo, the simulation of sensor data and the modeling of LLSF environment and the robots actions in this environment. Because we want to use the multi-robot simulation to evaluate and improve our own LLSF system, we will also develop some concrete improvements and compare the performances of the system with different configurations. The majority of these improvements relate to the high level agent, which is responsible for the decisions of the robot and the coordination between the agents.
\\
In chapter 2, we show the background of this thesis. This includes descriptions of the Robotino, the RoboCup and LLSF, Fawkes, Gazebo and the current LLSF solution of the Carologistics team. In chapter 3, we present related work about other simulations and agent strategies. The design ideas and thoughts behind the simulation are described in chapter 4. In chapter 5, we present the implementation. This includes general usage of the used tools, a description of the developed simulation modules and the improvements on our LLSF system. The evaluation-results of the agent changes and the simulation itself are shown in chapter 6. In chapter 7, we suggest future work and conclude in chapter 8.
